%%
%% Beamer presentation on Linux synchronisation mechanisms.
%% Copyright (c) 2012 by Michał Nazarewicz <mina86@mina86.com>.
%%
%% Distributed under the terms of the Creative Commons
%% Attribution-ShareAlike 3.0 Unported (CC BY-SA 3.0) license which
%% can be found at <http://creativecommons.org/licenses/by-sa/3.0/>.
%%
\documentclass[12pt]{beamer}
\usepackage{polski}
\usepackage[utf8]{inputenc}
\usepackage{listings}


\newcommand{\code}[2]{
  \begin{block}{#1}
    \lstinputlisting{code/#2.c}
  \end{block}
}

\newcommand{\codepart}[3]{
  \begin{block}{#1}
    \lstinputlisting[linerange=#3]{code/#2.c}
  \end{block}
}

\newcommand{\columnscode}[4]{
  \begin{block}{#1}
    \begin{columns}[c]
      \column{.45\textwidth}
      \lstinputlisting[linerange=#3]{code/#2.c}
      \column{.45\textwidth}
      \lstinputlisting[linerange=#4]{code/#2.c}
    \end{columns}
  \end{block}
}

\newcommand{\vv}[1]{\textcolor{blue}{#1}}
\newcommand{\sv}[1]{\vv{\small #1}}
\newcommand{\pv}[1]{\small (\vv{#1})}


\usetheme{Frankfurt}
\setbeamertemplate{headline}{%
  \begin{beamercolorbox}{section in head/foot}
    \vskip2pt\insertnavigation{\paperwidth}\vskip2pt
  \end{beamercolorbox}%
}
\setbeamertemplate{navigation symbols}{}


\lstset{
  language=C,
  emptylines=1, showlines=false,
  basicstyle=\scriptsize,
%  numbers=left, numberstyle=\scriptsize, numbersep=5pt, numberblanklines=false,
  columns=flexible,
  showstringspaces=false, tabsize=4,
  breaklines=true, breakatwhitespace=true, breakautoindent,
  literate=--1
}


\title{Synchronizacja w~jądrze Linux}
\author{Michał Nazarewicz}
\institute{Instytut Informatyki Politechniki Warszawskiej}


\begin{document}

\frame{\titlepage}

\frame{\tableofcontents}


\section{Wstęp} %%%%%%%%%%%%%%%%%%%%%%%%%%%%%%%%%%%%%%%%%%%%%%%%%%%%%%%%%%%%%%


\subsection{Po co synchronizować?}

\frame{
  \frametitle{Po co synchronizować?}

  \begin{itemize}
  \item Systemy z~wieloma procesorami stają się powszechne.
  \item Współbieżność pozwala lepiej wykorzystać zasoby sprzętu.
  \item Stosowanie wątków może uprościś logikę aplikacji.
    \begin{itemize}
    \item Jest to prawdziwe również w~jądrze.
    \end{itemize}
  \item Jeżeli wiele jednostek wykonania operuje na tych samym danych
    potrzebna jest synchronizacja.
  \end{itemize}

  \pause

  \begin{block}{Wyścig}
    Sytuacja, w której co najmniej dwa logiczne konkteksty wykonania
    (procesy, zadania itp.) wykonują operację na zasobach dzielonych,
    a ostateczny wynik zależy od momentu realizacji.

    \rightline{\small za dr.\ Krukiem}
  \end{block}
}

\frame{
  \frametitle{Po co synchronizować?, kont.}

  \begin{block}{Inkrementacja licznika}
    \begin{center}
      \begin{tabular}{l|l}
        {\bf Wątek 1} & {\bf Wątek 2} \\ \hline
        rejestr $\leftarrow$ licznik     & \\
        rejestr $\leftarrow$ rejestr + 1 & rejestr $\leftarrow$ licznik \\
        licznik $\leftarrow$ rejestr     & rejestr $\leftarrow$ rejestr + 1 \\
                                         & licznik $\leftarrow$ rejestr \\
      \end{tabular}
    \end{center}
  \end{block}

  \pause

  \begin{block}{Ogólny schemat odczyt-modyfikacja-zapis}
    \begin{center}
      \begin{tabular}{l|l}
        {\bf Wątek 1} & {\bf Wątek 2} \\ \hline
        odczyt      & \\
        modyfikacja & odczyt \\
        zapis       & modyfikacja \\
                    & zapis \\
      \end{tabular}
    \end{center}
  \end{block}
}


\subsection{Synchronizacja w~kontekście jądra?}

\frame{
  \frametitle{Synchronizacja w~kontekście jądra?}

  \begin{itemize}
  \item W~przestrzeni użytkownika też synchronizujemy.
  \item To już wszystko było!
  \item Ale\ldots
  \end{itemize}

  \begin{itemize}
  \item Każdy cykl zegara spędzony w~jądrze to cykl stracony.
  \item Jednocześnie wszyscy korzystają z~usług jądra.
  \item Stąd bardzo duży nacisk na wydajność.
  \end{itemize}

  \begin{itemize}
  \item Jądro pracuje w~wielu kontekstach.
  \item Przerwania przychodzą w~dowolnym momencie.
  \item Jest wiele poziomów przerwań.
  \end{itemize}
}


\section{Mechanizmy nieblokujące} %%%%%%%%%%%%%%%%%%%%%%%%%%%%%%%%%%%%%%%%%%%%

\frame{
  \frametitle{Mechanizmy nieblokujące}

  \begin{itemize}
  \item Tradycyjna sekcja krytyczna zmusza wątki na czekanie przy
    wchodzeniu.
  \item Czekanie to\ldots{} strata czasu.
  \item Lepiej nie czekać, niż czekać.
  \item Wiele algorytmów można zrealizować korzystając z~,,lekkich''
    mechanizmów synchronizacji, które nie blokują kontekstu wykonania.
  \end{itemize}
}


\subsection{Brak synchronizacji i~bariery pamięci}

\frame{
  \frametitle{Kiedy nie synchronizować?}

  \begin{itemize}
  \item Istnieje duży nacisk na wydajność jądra.
  \item Dlatego niektóre operacje nie są synchronizowane.
  \item System zakłada, że jest to powinność użytkownika.
  \end{itemize}
}

\frame{
  \frametitle{Kiedy nie synchronizować?, kod}
  \code{Wywołanie systemowe {\tt read(2)}}{read}
}

\frame{
  \frametitle{Bufor cykliczny}

  \begin{itemize}
  \item Jeden producent.
  \item Jeden konsument.
  \item Jak obsługa klawiatury w~DOS-ie.
  \end{itemize}
}

\frame{
  \frametitle{Bufor cykliczny, kod}

  \codepart{Bufor cykliczny, dane}{cyclic}{1-2}
  \columnscode{Bufor cykliczny, funkcje}{cyclic}{10-19}{20-29}
}

\frame{
  \frametitle{Bariery pamięci}

  \begin{itemize}
  \item Procesor może przestawiać operacje zapisu i~odczytu.
  \item Procesor może postrzegać operacje w~losowej kolejności.
  \item Bariery wymuszają częściowy porządek operacji.
  \end{itemize}

  \pause

  \columnscode{Bufor cykliczny, poprawiony}{cyclic}{30-39}{40-49}
}

\frame{
  \frametitle{Bariery pamięci, kont.}

  \code{{\bf Niepoprawny} wielowątkowy singleton}{invalid-singleton}
}


\subsection{Zmienne atomowe i~licznik referencji}

\frame{
  \frametitle{Zmienne atomowe}

  \begin{itemize}
  \item Linux wyposażony jest w~zmienne atomowe \pv{atomic\_t}.
  \item Operacja na zmiennych atomowych są\ldots{} atomowe.
  \end{itemize}

  \begin{block}{Dostępne operacje na zmiennych atomowych}
    \begin{center}
      \begin{tabular}{lll}
        \vv{atomic\_set} & \vv{atomic\_read} & \\
        \hline
        \vv{atomic\_add} & \vv{atomic\_add\_return} & \vv{atomic\_add\_negative} \\
        \vv{atomic\_sub} & \vv{atomic\_sub\_return} & \vv{atomic\_sub\_and\_test} \\
        \vv{atomic\_inc} & \vv{atomic\_inc\_return} & \vv{atomic\_inc\_and\_test} \\
        \vv{atomic\_dec} & \vv{atomic\_dec\_return} & \vv{atomic\_dec\_and\_test} \\
        \hline
        & \vv{atomic\_add\_unless} & \vv{atomic\_inc\_not\_zero} \\
        \hline
        & \vv{atomic\_xchg} & \vv{atomic\_cmpxchg} \\
      \end{tabular}
    \end{center}
  \end{block}
}

\frame{
  \frametitle{Licznik referencji}

  \begin{itemize}
  \item W~oparciu o~zmienne atomowe zaimplementowany jest licznik
    referencji \pv{struct kref}.
  \item Dostępne operacje:
    \begin{itemize}
    \item \vv{kref\_set},
    \item \vv{kref\_init},
    \item \vv{kref\_get} oraz
    \item \vv{kref\_put}.
    \end{itemize}
  \end{itemize}
}

\frame{
  \frametitle{Licznik referencji, kod}

  \columnscode{Obiekt współdzielony}{kref}{1-12}{14-27}
}

\frame{
  \lstset{basicstyle=\tiny}
  \code{Zmienne atomowe w~przestrzeni użytkownika}{string-ref}
  \lstset{basicstyle=\scriptsize}
}


\subsection{Operacje na bitach}

\frame{
  \frametitle{Operacje na bitach}

  \begin{itemize}
  \item Podobnym mechanizmem są atomowe operacja na bitach.
  \item Nadają się idealnie, jeżeli chcemy wykonać coś raz.
  \end{itemize}

  \begin{block}{Dostępne operacje na bitach}
    \begin{center}
      \begin{tabular}{ll}
        \vv{set\_bit} & \vv{test\_and\_set\_bit} \\
        \vv{clear\_bit} & \vv{test\_and\_clear\_bit} \\
        \vv{change\_bit} & \vv{test\_and\_change\_bit} \\
      \end{tabular}
    \end{center}
  \end{block}
}

\frame{
  \frametitle{Operacje na bitach, kod}

  \code{Tasklety}{tasklets}
}


\subsection{Porównaj i~zamień}

\frame{
  \frametitle{Porównaj i~zamień}

  \begin{block}{Atomowe operacje porównaj i~zamień}
    \begin{center}
      \begin{tabular}{ll}
        \vv{xchg}    & \vv{atomic\_xchg} \\
        \vv{cmpxchg} & \vv{atomic\_cmpxchg} \\
      \end{tabular}
    \end{center}
  \end{block}

  \pause

  \columnscode{Operacje na stosie}{stack}{1-9}{11-18}
}


\section{,,Śpiące'' mechanizmy blokujące} %%%%%%%%%%%%%%%%%%%%%%%%%%%%%%%%%%%%

\frame{
  \frametitle{Mechanizmy blokujące}

  \begin{itemize}
  \item Proste atomowe operacje często nie wystarczają.
  \item Zazwyczaj trzeba modyfikować wiele zmiennych ,,na raz''.
  \item Często trzeba również czekać na wystąpienie pewnych warunków.
  \item W~takich sytuacjach sprawdzają się mechanizmy blokujące.
  \end{itemize}
}


\subsection{Kolejka oczekiwania}

\frame{
  \frametitle{Kolejka oczekiwania}

  \begin{itemize}
  \item Podstawowym mechanizmem blokującym jest kolejka oczekiwania
    \pv{wait\_queue\_head\_t}, która
  \item służy do tworzenia listy, w~której zadania oczekują, aż
  \item inne zadania je obudzą.
  \item Zazwyczaj korzysta się z~niej poprzez interfejs
    \vv{wait\_event} (i~warianty).
  \end{itemize}
}

\frame{
  \frametitle{Kolejka oczekiwania, kod}

  \codepart{Blokujący bufor cykliczny, dane}{cyclic}{50-59}
  \columnscode{Blokujący bufor cykliczny, funkcje}{cyclic}{60-69}{70-79}
}

\frame{
  \frametitle{Spanie a~sygnały}

  \begin{itemize}
  \item Funkcja \vv{wait\_event} blokuje sygnały!
  \item Procesu blokującego sygnały nie da się zabić!
  \item Lepiej użyć \vv{wait\_event\_interruptible}.
  \end{itemize}

  \pause

  \columnscode{Kolejka oczekiwania, wersja {\tt
      interruptible}}{cyclic}{80-89}{90-99}
}


\subsection{Semafory i~muteksy}

\frame{
  \frametitle{Semafory}

  \begin{itemize}
  \item Użyty mechanizm to klasyczny semafor.
  \item Semafory są rzecz jasna dostępne w~jądrze.
  \item Dostępne operacje:
    \begin{itemize}
    \item \vv{sema\_init},
    \item \vv{down}, \vv{down\_trylock},
    \item \vv{down\_interruptible}, \vv{down\_killable}, \vv{down\_timeout} oraz
    \item \vv{up}.
    \end{itemize}
  \end{itemize}

  \begin{block}{Semafor}
    Obiekt inicjowany nieujemną liczbą całkowitą, na której zdefiniowane
    są dwie niepodzielne operacje:
    \begin{itemize}
    \item {\tt down(sem) \{ while (!sem); --sem; \}}.
    \item {\tt up(sem) \{ ++sem; \}} i
    \end{itemize}

    \rightline{\small za dr.\ Krukiem}
  \end{block}
}

\frame{
  \frametitle{Semafory, kod}

  \codepart{Bufor cykliczny na semaforach, dane}{cyclic}{100-109}
  \columnscode{Bufor cykliczny na semaforach, funkcje}{cyclic}{110-119}{120-129}
}

\frame{
  \frametitle{Muteksy}

  \begin{itemize}
  \item Pewnym szczególnym rodzajem semaforów są muteksy.
  \item Z~założenia stworzone do implementowania sekcji krytycznej.
  \item Dostępne operacje:
    \begin{itemize}
    \item \vv{mutex\_is\_locked},
    \item \vv{mutex\_lock} i~wiarianty
      \begin{itemize}
      \item plus wersje \vv{\ldots\_nested},
      \end{itemize}
    \item \vv{mutex\_try\_lock},
    \item \vv{mutex\_unlock}.
    \end{itemize}
  \end{itemize}
}

\frame{
  \frametitle{Muteksy, kod}

  \codepart{Kolejka chroniona muteksem}{queue}{10-29}
}


\frame{
  \frametitle{Muteksy, ,,nested''?}

  \begin{block}{Zakleszczenie}
    Pojęcie opisujące sytuację, w której co najmniej dwie różne akcje
    czekają na siebie nawzajem, więc żadna nie może się zakończyć.

    \rightline{\small za Wikipedią}
  \end{block}

  \pause

  \begin{itemize}
  \item Linux ma mechanizm weryfikacji synchronizacji.
  \item Dzieli muteksy itp.\ na klasy.
  \item Sprawdza, czy klasy są blokowane w~tej samej
    kolejności.
  \item Nie można zablokować dwóch muteksów z~tej samej klasy.
  \item Co w~takim razie z~muteksami w~hierarchii?
  \item Po to jest właśnie \vv{mutex\_lock\_nested} itp.
  \end{itemize}
}


\section{,,Nieśpiące'' mechanizmy blokujące} %%%%%%%%%%%%%%%%%%%%%%%%%%%%%%%%%

\frame{
  \frametitle{,,Nieśpiące'' mechanizmy}

  \begin{itemize}
  \item Przedstawione dotychczas mechanizmy blokujące mogą przełączyć
    zadanie w~stan uśpienia.
  \item W~pewnych kontekstach spanie jest niedopuszczalne.
    \begin{itemize}
    \item Wewnątrz funkcji obsługi przerwania.
    \item Gdy przerwania są wyłączone.
    \item Ogólnie w~kontekście atomowym.
    \end{itemize}
  \item Ale na szczęście są inne mechanizmy.
  \end{itemize}
}

\subsection{Przerwania i~wywłaszczanie}

\frame{
  \frametitle{Wyłączanie przerwań}

  \begin{itemize}
  \item Będąc w~kodzie jądra możemy wyłączyć przerwania.
  \item \vv{local\_irq\_diable} wyłącza, a~\vv{local\_irq\_enable}
    włącza przerwania.
  \item Ale co jeśli w~momencie wejścia do sekcji krytycznej
    przerwania już były wyłączone?
  \item Na pomoc przychodzą \vv{local\_irq\_save}
    i~\vv{local\_irq\_restore}.
  \item Trzeba uważać -- pewne funkcje mogą spowodować przełączenie
    kontekstu nawet jeżeli przerwania są wyłączone.
  \item A~poza tym, wyłączanie przerwań jest lokalne dla procesora.
  \end{itemize}
}

\frame{
  \frametitle{Wyłączanie wywłaszczania}

  \begin{itemize}
  \item Linux wspiera wywłaszczanie wewnątrz kodu jądra.
  \item Zmniejsza to czas reakcji, ale
  \item komplikuje synchronizację.
  \item Można ten mechanizm wyłączyć.
  \item Dostępne operacje \vv{preempt\_disable}
    i~\vv{preempt\_enable}
  \item są rekurencyjne,
  \item ale lokalne dla konkretnego procesora.
  \end{itemize}
}


\subsection{Spinlocki}

\frame{
  \frametitle{Spinlock}

  \begin{itemize}
  \item Spinlock działa jak muteks, ale
  \item stosuje aktywne oczekiwanie, dzięki czemu
  \item nie zasypia.
  \item Użycie: \vv{spin\_lock} \ldots{} \vv{spin\_unlock}.
    \begin{itemize}
    \item Też mają wersję \vv{\ldots\_nested}.
    \end{itemize}
  \end{itemize}
}

\frame{
  \frametitle{Spinlock, kod}

  \codepart{Kolejka chroniona spinlockiem}{queue}{30-49}
}

\frame{
  \frametitle{Spinlock a~przerwania}

  \begin{itemize}
  \item Co jeśli w~sekcji krytycznej przyjdzie przerwanie?
  \end{itemize}

  \begin{block}{Przerwanie w~sekcji krytycznej}
    \begin{center}
      \begin{tabular}{lll}
        {\bf Kontekst} & & \\
        {\bf użytkownika} & & {\bf Przerwanie} \\
        \hline
        \vv{spin\_lock} & & \\
        & przychodzi przerwanie & \\
        & & \vv{spin\_lock} \\
        & & zakleszczenie
      \end{tabular}
    \end{center}
  \end{block}

  \begin{itemize}
  \item Gdy spinlock używany jest również w~przerwaniu, w~kontekście
    użytkownika trzeba przerwania wyłączyć {\small
      (\vv{spin\_lock\_irqsave} \ldots{}
      \vv{spin\_unlock\_irqrestore})}.
    \begin{itemize}
    \item Jest też wersja \vv{spin\_lock\_irq} \ldots{}
      \vv{spin\_unlock\_irq}
    \end{itemize}
  \end{itemize}
}

\frame{
  \frametitle{Spinlock a~przerwania, kod}

  \codepart{Kolejka chroniona spinlockiem, poprawiona}{queue}{50-69}
}


\section{Zakończenie} %%%%%%%%%%%%%%%%%%%%%%%%%%%%%%%%%%%%%%%%%%%%%%%%%%%%%%%%

\subsection{Podsumowanie}

\frame{
  \frametitle{Podsumowanie}

  \begin{itemize}
  \item Linux jest bardzo złożonym oprogramowaniem.
  \item Wiele mechanizmów jest znanych z~przestrzeni użytkownika.
  \item Nacisk na wydajność i~przerwania komplikują sytuację.
  \item Dlatego jest wiele mechanizmów unikalnych dla jądra.
  \item Czyni to synchronizację w~kernelu bardzo ciekawą!
  \item Ale i~złożoną i~trudną.
  \end{itemize}
}

\frame{
  \frametitle{Czego nie było\ldots}

  \begin{itemize}
  \item Siedem rodzai barier pamięci.
  \item Big Kernel Lock.
  \item Kolejka oczekiwań to coś o~wiele więcej niż \vv{wait\_event}.
  \item \vv{spin\_lock\_bh}.
  \item Zmienne {\it per CPU}.
  \item Alokacja pamięci a~synchronizacja.
  \end{itemize}
}

\subsection{Źródła}

\frame{
  \frametitle{Skąd czerpać informacje?}

  \begin{itemize}
  \item Use the Source, Luke.
  \item Podkatalog {\tt Documentation} w~źródlach.
  \item Publiczne serwery LXR.
    \begin{itemize}
    \item http://lxr.linux.no/
    \end{itemize}
  \item LWN.net
  \item STER z dr. W. Zabołotnym
  \item {\it Linux Device Drivers, Third Edition}
    \begin{itemize}
    \item http://lwn.net/Kernel/LDD3/
    \end{itemize}
  \item Daniel P. Bovet, Marco Cesati. {\it Understanding the Linux
    Kernel}.
  \end{itemize}
}

\frame{
  \frametitle{Dziękuję za uwagę!}

  \begin{itemize}
  \item Pytania?
  \item Opinie?
  \item Sugestie?
  \end{itemize}

  \begin{itemize}
  \item \insertauthor
  \item http://mina86.com/
  \item mina86@mina86.com
  \item mina86@jabber.org
  \end{itemize}
}

\end{document}
